\section{Introduction}
The number of processor architectures has grown significantly in the recent
years. We saw the appearance of accelerators such a GPUs as a response to the
many obstacles met by processor architects, e.g. the frequency and memory walls.
GPUs are application specific architectures, well suited for data parallelism,
for code with regular memory access patterns such as those available in
numerical or graphics applications. Compared to CPUs, GPUs contain simpler cores
but compensate with their higher number (tens of cores compared to 4 - 8 for
CPUs). The cores operate at a lower frequency (often 1 GHz against 2 - 3 GHz for
CPUs) and are in-order, hiding the latency of the instruction pipeline by
interleaving on each core the execution of thousands of threads.

A simple abstraction for GPUs is to consider them many-core architectures, with
a large number of cores and with each core including large vector units
allowing to compute up to 32 single precision flops per cycle (64 flops for
FMAD). Compute Unified Device Architecture (CUDA) is an option for programming
Nvidia GPUs. The vector units are not directly visible in CUDA. Indirectly, the
SIMD nature of GPUs reveals itself when applying various optimizations, e.g. for
bank conflicts and memory access coalescing \cite{cuda}. This view of GPUs as
vector processors is at the foundation of
\cite{Volkov:2008:BGT:1413370.1413402}. The tuning strategy presented there was
afterwards included in the \textit{cublas} library from Nvidia.

% CUDA can be misleading when referring to the architecture. Of course, a
% trade-off should be clarified at this point. It does what it does to simplify
% programming with the cost of hiding the real hardware. Through CUDA lenses, the
% GPU is an array of hundreds of scalar cores which for us they generally map to
% SIMD lanes. The cores are at the immediate upper level contained in a Streaming
% Multi-processor (SM). For Nvidia Fermi GPUs, an SM contains 32 of such cores. SM
% has low latency, local memory which is reconfigurable allowing for different
% ratios between the scratchpad and the L1 cache. At a upper layer we have an 768
% KB L2 cache improving the access to the RAM on the GPU. This is the hardware
% part. On top of it, CUDA defines how the software maps to it. The main concept
% here is the thread. The threads are implicitly executed as warps of 32, each
% branch taken by one of the 32 reduces performance, much as it happens when
% masking the lanes of a SIMD unit. CUDA threads are grouped explicitly this time
% in blocks and a block is executed on an SM. Communication among threads is
% allowed only within a block.

But when reasoning about code transformations that improve performance on a wide
range of architectures, we do not want to lose ourselves in the details of the
programming models. This high-level simplified view allows us to have more
flexibility when adapting the code to new architectures. However, this comes at
a cost of hiding some features that would otherwise increase the performance.
Considering this trade-off, our purpose is therefore to develop high-level code
transformations that address common features among different devices. We do not
expect that the sequence of transformations remains the same. To simplify the
porting of applications from one architecture to another, it is essential to
determine the common and different aspects of the architectures and map the
optimizations accordingly. As an example, CPUs and GPUs have vector units and
Array-Of-Structures (AOS) to Structure-Of-Arrays (SOA) is a transformation that
often is required by both. Given the convergence of CPUs and GPUs seen in CPUs
that have doubled the width of the SIMD units and GPUs that received 2 level,
coherent caches \cite{fermi}, we expect that more transformations, e.g. for
locality, to be shared between codes for CPUs and GPUs.

To test these concepts, we consider a computational steering application. The
main goal in this application is to allow for a smooth, real-time interaction
with compressed simulation data. We have two phases (or algorithms) being at the
core of the compression and decompression of the simulation data. These
algorithms are in fact based on a numerical method called the sparse grid
technique which allows for a hierarchical and efficient representation of
high-dimensional functions. Using it, it was shown that sheer amounts of
simulation data can be managed efficiently both with regard to time and space,
i.e. memory footprint.

For enhanced performance, it was important to benefit from the latest
developments in hardware. But starting from scratch with every hardware release
is far from being a productive approach. Instead, what we followed was to
develop optimizations for loops and data layout at the same level at which we
develop algorithms. We realized that many of the transformations we are applying
are valid for both CPUs and GPUs, considering the above mentioned view of GPUs
as a vector processors. Moreover, we consider that our study is important giving
that more and more programming models are emerging and the open question is if
we can design optimizations that have performance portability. The main
contributions of our paper are as follows:

\begin{itemize}
  \item We port dimensionally adaptive sparse grids algorithms on GPUs and
  report a speedup of 28.3x compared to the CPU versions.
  \item We present a study on the performance of a subset of common loop
  transformations for two different architectures, CPUs and GPUs. We describe
  the challenges met along the way and their solutions.
  \item We report speedup results that, for the considered computational
  steering application, accelerate the state-of-the-art implementation up to a
  factor of 6.2x.
\end{itemize}


