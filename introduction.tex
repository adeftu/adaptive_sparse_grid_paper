\section{Introduction}
The number of processor architectures has grown significantly in the recent
years. We saw the appearance of accelerators such a GPUs as a response to the
many obstacles met by processor architects, e.g. the frequency and memory walls.
GPUs are application specific architectures, well suited for data parallelism,
for code with regular memory access patterns such as those available in
numerical or graphics applications. Compared to CPUs, GPUs contain simpler cores
but compensate with their higher number (tens of cores compared to 4 - 8 for
CPUs). The cores operate at a lower frequency (often 1 GHz against 2 - 3 GHz for
CPUs) and are in-order, hiding the latency of the instruction pipeline by
interleaving on each core the execution of thousands of threads.

An abstraction of GPUs is to see them as many-core architectures, with a large
number of cores and with each core including large vector units allowing  o
compute up to 32 single precision floating point operations per cycle (64 flops
if we count FMAD as 2 operations, i.e. a multiply and an addition). Nvidia GPUs
can be programmed through the well known CUDA framework. Although the vector
units are not directly visible in CUDA, the SIMD nature of GPUs reveals itself
when applying various optimizations, e.g. for bank conflicts and memory access
coalescing \cite{cuda}. Viewing GPUs as vector processors is supported by a very
efficient implementation of dense linear algebra operations on GPUs described in
\cite{Volkov:2008:BGT:1413370.1413402}.

When reasoning about code transformations that improve performance on a wide
range of architectures, we do not want to lose ourselves in the details of the
programming models since this often results in lower programming productivity.
A more abstract view of the hardware focusing on hardware characteristics rather
than programming models allows for more flexibility when adapting a code to a
new architecture. For improved productivity and performance, our goal is to
develop high-level code transformations that address characteristics common
across different processors, namely CPUs and GPUs. In order to simplify the
porting of an application to a new architecture, it is important to determine
the similarities and differences between the old and the new architecture and
map the optimizations accordingly. In defense of this statement, CPUs and GPUs
both have vector units and array-of-structures to structure-of-arrays (AOS to
SOA) is a transformation that often is required by both. Given the current
convergence trend seen in CPUs and GPUs, i.e. CPUs have doubled the width of the
SIMD units from SSE to AVX and GPUs have been equiped with a 2 level cache
memory \cite{fermi}, we expect that increasingly more transformations, e.g. for
locality, to be shared between codes for CPUs and GPUs.

In order to support our ideas, we consider a computational steering application
whose main goal is to provide real-time interaction with compressed simulation
data \cite{Butnaru201156}. The two hotspots of this application are the routines
implementing two advanced algorithms for data compression and decompression.
Data compression is in this case achieved through a a numerical method called
the \textit{sparse grid technique} which is similar to wavelets to some extent.
More exactly, sparse grids allow for a hierarchical and memory efficient
representation of high-dimensional simulation data. Note that sparse grids must
not be confused with sparse matrices mainly because of the high-dimensional
settings in which sparse grids are used, e.g. 10-dimensional, as opposed to
matrices which are 2-dimensional.

In order to reduce the response time of our computational steering application,
it is important to leverage the latest hardware. However, restarting the
optimization process with every major hardware release can be very
counter-productive and therefore should be avoided. Instead, what we follow is
to develop optimizations for loops and data layout at the same abstraction level
at which we develop algorithms. The main conclusion supported by this paper is
that most of the transformations proposed for our test application are
applicable and beneficial to both CPUs and GPUs. Our optimization approach is
relies highly on an analysis of CPUs and GPUs based on similarities and
differences, putting minimal emphasis on the programming model which in our view
can actually complicate the porting process. Moreover, we believe that our study
is important given the fact that increasingly more programming models are
emerging and the open question is whether or not programmers can design
optimizations portable across different types of processors, e.g. CPUs and GPUs.

The main contributions of our paper are as follows:

\begin{itemize}
  \item We port for the first time the sparse grid library \textit{fastsg} on GPUs and
  report a speedup of 28.3x compared to the optimized CPU version.
  \item We study the performance benefits of a set of loop transformations on CPUs and GPUs. 
  We describe the challenges met along the way and their solutions.
  \item We report speedup numbers in the context of our computational steering application
  accelerate the state-of-the-art implementation up to a factor of 6.2x for compression
  and 1.5x for decompression on GPUs.
\end{itemize}

The rest of the paper is structured as follows. In
Section~\ref{sec:related_work} we place our work in the field code optimizations
and sparse grids, and relate to the published work.
Section~\ref{sec:cpus_vs_gpus} highlights the similarities between CPUs and
GPUs, allowing for a reuse of optimizations across architectures.
Section~\ref{sec:comp_steering} introduces sparse grids and describes the core
functionality of the sparse grid library \textit{fastsg}. In
Section~\ref{sec:optimizations} we describe a set of loop transformations
suitable for both architectures. Section~\ref{sec:op_strategies} shows our
optimizations for CPUs and GPUs based on the aforementioned ideas. In
Section~\ref{sec:evaluation} we present our results and show that despite using
different architectures, similarities in terms of hardware characteristics have
the potential to easen code porting. Finally, Section~\ref{sec:conclusion}
concludes the paper.
