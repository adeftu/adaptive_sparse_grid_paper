\section{Introduction}
The number of processor architectures has grown significantly in the recent years. We saw the appearance of accelerators such a GPUs as a response to the many obstacles met by processor architects, e.g. the frequency and memory walls. GPUs are application specific architectures, well suited for data parallelism, for codes with regular memory access patterns such as those available in numerical or graphics applications. Compared to CPUs, GPUs contain simpler cores but compensate with their higher number (tens of cores compared to 4 - 8 for CPUs). The cores operate at a lower frequency (often 1 GHz against 2 - 3 GHz for CPUs) and are in-order, hiding the lantency of the instruction pipeline by interleaving on each core the execution of thousands of threads.

A simple abstraction for GPUs is to consider them many-core architectures, with a large number of cores and with each core including a large vector units allowing to compute up to 32 flops per cycle (64 flops for FMAD). This view of GPUs is not immediately visible in the programming manuals released by GPU vendors. CUDA is an option for programming Nvidia GPUs. The vector units are not directly visible in CUDA. Indirectly, the SIMD nature of GPUs reveals itself when applying optimizations for bank conflicts and memory access coalescing (see CUDA manual for more details). Seeing the GPUs as vector processors is at the foundation of Volkov's paper from SC 2008 on optimizing matrix multiplication. His tuning strategy was afterwards included in the cublas library from Nvidia.

CUDA can be misleading when refering to the architecture. Of course, a tradeoff should be clarified at this point. It does what it does to simplify programming with the cost of hiding the real hardware. Through CUDA lenses, the GPU is an array of hundreds of scalar cores which for us they generally map to SIMD lanes. The cores are at the immediate upper level contained in a Streaming Multi-processor (SM). For Nvidia Fermi GPUs, an SM contains 32 of such cores. SM has low latency, local memory which is reconfigurable allowing for different ratios between the scratchpad and the L1 cache. At a upper layer we have an 768 KB L2 cache improving the access to the RAM on the GPU. This is the hardware part. On top of it, CUDA defines how the software maps to it. The main concept here is the thread. The threads are implicitly executed as warps of 32, each branch taken by one of the 32 reduces performance, much as it happens when masking the lanes of a SIMD unit. CUDA threads are grouped explicitly this time in blocks and a block is executed on an SM. Communication among threads is allowed only within a block.

When reasoning about code transformations that improve performance on a wide range of architectures, we do not want to lose ourselves in the details of the programming models. We need the real view of the hardware not a simplified view. Ideally, what we want is to develop high-level code transformations for which it can be decided at design time performance improvement they give on a given architecture, i.e. CPUs or GPUs. We do not expect that the sequence of transformations remains the same. To simplify the porting of applications from one architecture to another, it is essential to determine the common / different points of the architectures and map the optimizations accordingly. As an example, CPUs and GPUs have vector units and AOS to SOA is a transformation that often is required by both. Given the converge of CPUs and GPUs seen in CPUs that have doubled the width of the SIMD units (Intel Sandy Bridge has 2x the bits of Intel Nehalem, 128 bits) and GPUs that received 2 level, coherent caches, we expect that more transformations, e.g. for locality, to be shared between codes for CPUs and GPUs.

To test these concepts, we consider a computational steering application. The main goal in this application is allow for a smooth, real-time interaction with compressed simulation data. We have three phases (or algorithms) being at the core of the compression and decompression of the simulation data. Decreasing the execution of the routines decreases proportionally the response time of our visualization system. These algorithms are in fact based on a numerical method called the sparse grid technique which allows for a hierarchical and efficient representation of high-dimensional functions. Using it, it was shown that sheer amounts of simulation data can be managed efficienctly both with regard to time and space, i.e. memory footprint.

For enhanced performance, it was important to benefit from the latest developments in hardware. But starting from scratch with every hardware release is far from being a productive approach. Instead, what we followed was to develop optimizations for loops and data layout at the same level at which we develop algorithms. We realized that many of the transformations we are applying are valid for both CPUs and GPUs, considering the above view of GPUs as a vector processors. The main contributions of our paper are as follows:

\begin{itemize}
    \item We show our experiences with porting a number of ??? loop transformations to two different architectures, CPUs and GPUs. We describe the challenges met along the way and their solutions.
    \item We report speedup results that, for the considered computational steering application, accelerate the state-of-the art implementation up to a factor of ???.
\end{itemize}


