\section{Loop Transformations in \textit{fastsg}}
\label{sec:optimizations}

The majority of optimizations applied to numerical codes focus on loops since
these are the places where most of the execution time is spent. In general, we
expect the compiler to handle loop transformations. An overview of loop
transformations performed by the compiler is presented in
\cite{Bacon:1994:CTH:197405.197406}. We cover in our paper only a subset of
those optimizations relevant for our discussion and our routines. Specifically,
we focus on optimizations that improve the latency for memory access by
exploiting the deep memory hierarchy of modern processors, optimizations that
decrease the number of instructions executed and help vectorization, and those
that speed up arithmetic operations with integers.

\subsection{Memory Locality}

The first category includes transformations which improve memory locality. As
pointed out in Section~\ref{sec:cpus_vs_gpus}, GPUs have started to borrow
caches and scratchpad memories from CPUs, making them susceptible to a plethora
of new optimizations. \textit{Loop tiling} is a loop reordering transformation
primarily used to improve cache reuse by dividing the iteration space into
tiles. The transformation is highly dependent on the memory hierarchy. For GPUs
the tile can be chosen such that it fits into the cache of the microprocessor,
allowing the assignment of one or more tiles to each microprocessor. Aggressive
tiling is done for all memory levels, including registers, L1 and L2 caches,
etc. The complexity of caches complicates compiler's task to produce efficient
loop tiling transformations. An example of application for this kind of
transformation on GPUs is stencil codes \cite{volkov2010}, which was first
developed for CPUs. In the same category of loop transformations that enhance
locality, there is also \textit{loop interchange}, which exchanges the position
of two loops so that the access to memory is improved, i.e. fewer cache and TLB
misses. Sometimes, because the number of iterations of a loop is not known at
compile time, the compiler cannot determine the best permutation of loops.
Complex data dependencies resulting from complex array subscripts can also limit
the applicability of this optimization since the compiler needs to prove that a
given optimization preserves the semantics of the unoptimized code. A study
about automatic loop interchange on GPUs is presented in
\cite{Leung:2009:APG:1596655.1596670}.

\subsection{Vectorization}

In the recent period, \textit{vectorization} has been gaining increased
attention from developers. The current generation of processors from Intel,
Sandy Bridge, contains 8-lane SIMD units (AVX), providing two times more
throughput than the previous 4-lane SIMD units (SSE). In theory, automatic
speedup can be achieved through vectorization performed by compilers. However,
there are several factors that limit compilers' ability to vectorize a loop: 
data dependencies, aliasing, indexing arrays with complex expressions,
non-sequential access patterns, etc. Furthermore, vectorization is usually
applied after other loop transformations. Hence, if the compiler does not
reorder the loops, the innermost loop cannot be vectorized. We refer the reader
to \cite{vec_guide} for a complete vectorization guide on Intel processors 
and conclude that manual intervention is sometimes required to cope with
such inhibitors.

Among loop transformations that help compilers, there are loop unrolling, loop
unroll-and-jam, and software pipelining. \textit{Loop unrolling} replicates the
body of a loop a number of times referred to as the unrolling factor. This
optimization also reduces the loop overhead and can improve the register, cache,
and TLB locality. Another advantage is that more possibilities for reordering
instructions are available, thus resulting in a better utilization of the
instruction pipeline. The downside is an increase in size of the generated code
which puts more pressure on the instruction cache. \textit{Loop unroll-and-jam}
is similar to unroll in the sense that it copies the body of one of the next
inner loops some number of times equal to the unroll-and-jam factor. The
generated inner loops are then fused together, or jammed. This transformation
can lead to more independent instructions in the inner loop. Both loop unrolling
and loop unroll-and-jam pave the way for instruction level parallelism, but on
GPUs this also translates to a higher register usage, which in turn leads to a
reduced microprocessor occupancy. These optimizations combined with various
other techniques proved successful when porting some kernels from linear
algebra, like matrix multiplication \cite{Volkov:2008:BGT:1413370.1413402} or
symmetric matrix vector product (SYMV) \cite{Nath:2011:OSD:2063384.2063392}.
\textit{Software pipelining} implies reordering instructions from different
iterations of a loop such that independent instructions are executed in
sequence. It is often used in combination with loop unrolling.

\subsection{Integer Operations}

\textit{Strength reduction} involves replacing computationally expensive
operations with arithmetically equivalent, but less expensive ones. In our
particular case, instead of using divisions inside the loop body, we precompute
the inverses outside the loop and do multiplications instead. Another important
optimization in this category is \textit{loop invariant code motion}. This is a
data-flow based loop transformation that moves outside a loop the computation
whose result does not modify between iterations. The obvious benefit is a
decrease in the number of instructions executed.