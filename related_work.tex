\section{Related Work}
\label{sec:related_work}
Sparse grids are known in mathematics for their ability to tackle high-dimensional problems which suffer from the so-called ``curse of dimensionality'', i.e. the exponential dependency of the number of grid points on the number of dimensions. The sparse grid theory and applications are shown in \cite{CambridgeJournals:227245}.

A compact data structure with minimal memory consumption and efficient
algorithms for regular sparse grids have recently been presented in
\cite{Murarasu:2011:CDS:1941553.1941559}. Further support for dimensionally
truncated sparse grids on CPUs was added in \cite{murarasu12fastsg:}, giving the
flexibility to treat dimensions differently, by allowing more or less
discretization points per dimensions, and thus to control the refinement level on a per dimension
basis. In our paper, our starting point is given by these algorithms for dimensionally
truncated sparse grids. We propose transformations that allows us to port them efficiently on GPUs. In addition, we present
a set of loop transformations for increasing the performance on both CPUs and GPUs.

Our work is built on the foundation of optimization techniques for heterogeneous systems.
applied in the context of different applications. 
A set of techniques for efficient programming of hybrid systems in the context
of dense linear algebra were presented in \cite{Tomov:2010:TDL:1805333.1805388}.
Here they show how the computation can be split for a better exploitation of
each device, but do not address the problem of programming CPUs and GPUs using a
set of common features. Also related to this field of study is the work from
\cite{Augonnet:2011:SUP:1951453.1951454}, which shows techniques for efficiently
exploiting heterogeneous systems through a uniform execution model.

