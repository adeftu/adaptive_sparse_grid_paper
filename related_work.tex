\section{Related Work}
\label{sec:related_work}
Sparse grids are a special type of grids known for their ability to address
high-dimensional problems which suffer from the so-called ``curse of
dimensionality'', i.e. the exponential dependency of the number of grid points
on the number of dimensions. Sparse grid applications include solving PDEs, data mining
and recently computational steering. For the detailed theory and applicability the 
reader is referred to \cite{CambridgeJournals:227245}.

A data structure with minimal memory consumption and efficient
algorithms for sparse grids are described in 
\cite{Murarasu:2011:CDS:1941553.1941559}. Further functionality, i.e. for dimensionally
truncated sparse grids, and an efficient implementation on CPUs is covered by \cite{murarasu12fastsg:}.
Dimensionally truncated sparse grids give the flexibility to treat dimensions differently, 
by allowing more or less discretization points per dimension, and thus to control the refinement level of the grid
on a per dimension basis. Compared to previous work, we propose loop transformations that allows us to 
efficiently port dimensionally truncated sparse grids on GPUs. To the best of the authors' knowledge this is the first
GPU implementation of dimensionally truncated sparse grids. Moreover, emphasis is put on exposing a set 
of loop transformations that increase the performance on both CPUs and GPUs.

Our work is inspired by optimization techniques for heterogeneous systems
applied to several other applications. A set of techniques for
efficient programming of heterogeneous systems for dense linear algebra
are presented in \cite{Tomov:2010:TDL:1805333.1805388}. The authors show there how the computation 
can be divided across different processor types in order to fully use the system, but do not address 
the problem of programming CPUs and GPUs based on leveraging their similarities.
% an similarity oriented analysis as in our work. 
Related to same field is \cite{Augonnet:2011:SUP:1951453.1951454} which shows techniques for efficiently
exploiting heterogeneous systems through a uniform execution model.

