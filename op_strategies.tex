\section{Optimization Strategies}
\label{sec:op_strategies}

The computational core of \textit{fastsg} resides in \textit{hierarchize} and
\textit{evaluate} routines as presented in \cite{}. We will not enter here into
the details of these procedures because a summary is already available in
\cite{murarasu2011}. Instead, we present a set of optimizations for both of them
in the context of CPUs and GPUs. We group these optimizations into 3 categories
based on their end purpose: memory locality, vectorization, and operations with
integers.

\begin{algorithm}[tbp]
\small{
	\caption{Hierarchization.}
 	\label{alg:hierarchization}                       

 	\begin{algorithmic}[1]
 		\For{$t=1$ to $D$}
 			\For{$g=L$ downto $1$}
 				\State \Call{HierarchizeKernel}{$t, g$}
 			\EndFor
 		\EndFor
 		\algstore{bkbreak}
	\end{algorithmic}
 
	\begin{algorithmic}[1]
  		\algrestore{bkbreak}
  		\Procedure{HierarchizeKernel}{$t, g$}
	 		\For{$j \leftarrow \Call{BlockStart}{g} + lane; j < \Call{BlockEnd}{g}; j\leftarrow j + 32$}
	 			\If{$lane = 0$}
	 				\State $l \leftarrow idx2l(j)$
	 			\EndIf
	 			\State $i \leftarrow idx2i(j)$
	 			\State $(\textit{ll}, \textit{il}) \leftarrow \text{left}(l, i, t)$
	 			\State $(\textit{lr}, \textit{ir}) \leftarrow \text{right}(l, i, t)$
	 			\State $\textit{v1} \leftarrow \textit{asg1d}[\text{agp2idx}(\textit{ll},
	 			\textit{il})]$ \State $\textit{v2} \leftarrow
	 			\textit{asg1d}[\text{agp2idx}(\textit{lr}, \textit{ir})]$ \State$\textit{asg1d}[j] \leftarrow \textit{asg1d}[j] - (\textit{v1} +\textit{v2}) \cdot 0.5$
			\EndFor
		\EndProcedure
	\end{algorithmic}
}
\end{algorithm}

We start with a basic version of hierarchization algorithm for GPU where we loop
with $t$ variable over all dimensions, with $g$ variable over all \textit{groups}
and then we call the kernel procedure. Each warp on the GPU computes the
coefficients for one \textit{block} of the sparse grid in order to achieve a
better cache locality and to reduce the thread divergence within warps. Because
the coefficients for one \textit{block} are stored contiguously in memory, we
also want for these values to be accessed simultaneously. Therefore, each thread
within a warp starts reading at index equal to its lane and then subsequently
every element found at a distance multiple of 32 (the number of threads within a
warp). In this way, a warp will access the memory contiguously at each read
instruction with an increased throughput. Here $l$ and $i$ are the
\textit{levels} and respectively \textit{index} vectors which represent the
coordinates of one point in the grid. We store them in the shared memory because
they are frequently used. The routines \textit{left} and \textit{right} return
the left and respectively right parents (\textit{blocks} that the current one
depends on) of a point in dimension $t$, while \textit{agp2idx} returns the
index where point $(l, i)$ is stored in the 1d representation of the sparse
grid, $asg1d$.

We now optimize this basic algorithm and we start with optimizations for integer
operations. The first observation we make is that computing the $l$ vector for
each point in the \textit{block} is redundant because within a \textit{block},
it has the same value. Therefore loop invariant code motion optimization
(\textit{inv1}) moves the computation of $l$ outside the loop. 

Next, in \textit{inv2} we can compute for each \textit{block} of norm $g$ the
indices in $asg1d$ of all $g-1$ parent \textit{blocks} before the innermost
loop. We store them in an array (in the heap space in the case of CPU
implementation and in the shared memory of a CUDA multiprocessor for the GPU
version). To get the values for these parent points, we can now use this array
to index $asg1d$ and get the \textit{block}. What we still need is the index
inside the \textit{block}. This is computed inside the $j$ loop by converting
$i$ to the corresponding index.

\textit{inv3} starts from the observation that $\tilde{l}$ and $\tilde{i}$, the
\textit{levels} and respectively \textit{index} vectors of the parents, differ
from $l$ and $i$ of the current point only in one dimension. Therefore, the
conversion mentioned above is not necessary anymore, thus reducing the
complexity from $O(D)$ to $O(1)$.

Now, in the innermost loop over the current \textit{block}, what is left to
compute is the conversion $(l(t), i(t)) \rightarrow (\tilde{l}(t),
\tilde{i}(t))$ which has a complexity of $O(l(t))$. \textit{inv4} precomputes
the results of these conversions before the kernel launch and stores them in an
array. The advantage is that the complexity of the innermost loop is now $O(1)$,
but at a cost of higher memory footprint.

Lastly, in order to improve memory locality and the data access patterns,
\textit{ichg1} optimization does a loop interchange by moving the loop over the
\textit{groups} from line 2 inside the kernel. In this way, one warp will not
only process one \textit{block}, but also all its parents.

% We start with the algorithm for hierarchization. In this algorithm,
% \textit{asg1d} is a 1d array containing the values of a dimensionally adaptive
% sparse grid. We traverse \textit{asg1d} $d$ times, each time updating the value
% at a point based on the values of its dependencies, i.e. its parents, in the
% current dimension, $t$. The iterations of the $t$ loop
% % \footnote{We refer to loops using their respective loop counters.} 
% are dependent. The $g$ loop traverses \textit{groups} of grid points whereas the
% $b$ loop iterates over \textit{blocks} from the same group. Finally, the $k$
% loop is equivalent to the traversal of all the points of a \textit{block}.
% 
% The left and right parents of a point in dimension $t$ is returned by the
% routines \textit{left} and \textit{right} respectively. Computing the left
% parent in dimension $t$ is equivalent to solving the equation $(i_t - 1) \cdot
% 2^{-l_t} = \textit{il}_t \cdot 2^{-\textit{ll}_t}$, where $\textit{il}_t \in
% \{1, \dots, 2^{\textit{ll}_t+1}\}$ and $\textit{il}_t$ odd. Except for the
% $t$-th components, all the other components of $\textit{ll}$ and $\textit{il}$
% are equals to the ones of $l$ and $i$ respectively. Similarly, the right parent
% is given by $(i_t + 1) \cdot 2^{-l_t} = \textit{ir}_t \cdot 2^{-\textit{lr}_t}$.
% All its other components are the same as in $l$ and $i$. An important
% observation is that for both the left and right parent $\textit{ll}_t < l_t$ and
% $\textit{lr}_t < l_t$.
% 
% Understanding the parent concept is essential for understanding the cache
% locality of this algorithm. We first need to look at dependencies from a higher
% level. Remember that in our data layout, we pack the grids points in
% \textit{blocks}. The identifier of a \textit{block} is the $l$ shared by the
% points it contains. For any \textit{block} with the norm of its identifier $b$,
% the parents of its points reside in $b - 1$ blocks with the same components of
% their respective identifiers except the $t$-th. We can therefore say that the
% dependencies of the \textit{block} reside in a relatively compact memory space
% with a total size of $2^b - 1$, where $2^b$ is the size of the \textit{block}.
% This result invalidates part of the analysis presented in \cite{murarasu2011}
% that presents \textit{hierarchization} as a cache unfriendly operation. Note
% that the cache optimization is a direct benefit of the storage scheme.
% 
% In Alg.~\ref{alg:hierarchization} a considerable amount of time is spent
% computing the bijections, \textit{idx2agp} and \textit{agp2idx}. Our first two
% optimizations, \textit{opt1} and \textit{opt2}, target exactly this problem.
% They perform loop invariant code motion, placing operations from the bijections
% immediately before the innermost loop. \textit{opt1} is based on the observation
% that all the points in the same \textit{block} have the same $l$. Hence $l$ can
% be computed only once for any \textit{block} with $2^g$ points. This means it is
% safe to move the computation of $l$ from line 6 outside the $k$ loop.
% \textit{opt2} relies on the fact that a \textit{block} whose identifier has the
% norm $g$ has $g - 1$ \textit{block} dependencies. This means that we can compute
% the indices in \textit{asg1d} of these \textit{blocks} immediately before the
% innermost loop and store them in a small 1d array, $\textit{idx12mem}$ of size
% $n$. By doing so, we reduce the number of operations in lines 9 and 10. Remember
% that the bijection computes three indices. With this optimization only
% $\textit{idx3}$ has to be computed in the body of the $k$ loop. The sum
% $\textit{idx1} + \textit{idx2}$ is obtained by indexing $\textit{idx12mem}$
% using the $t$-th component of $\textit{ll}$ in line 9 and $\textit{lr}$ in line
% 10. In Sec.~\ref{sec:evaluation} we show that we cannot obtain this optimization
% automatically from compilers. Summarizing the results, we can say that
% \textit{opt1} provides a speedup of 1.75 over the basic (unoptimized) version.
% \textit{opt1} and \textit{opt2} accelerate together the routine approximately
% 3.75 times.
% 
% Our third optimization, \textit{opt3}, reduces the number of instructions
% executed in line 6. After moving the computation of $l$ outside the loop, in
% line 6 there is only $i$ to compute. This is the inverse of the operation that
% computes $\textit{idx3}$ (Formula~\ref{eq:idx3}). The complexity of this
% operation is \Otime{d}. We replace it with an iterator that generates the
% current $i$ based on the previous $i$. The central concept of the iterator is
% the following. It increments the values of $i$ starting with the last component
% until no carry is generated or there are no more components left. A carry is
% produced when the $t$-th component of $i$ reaches the value $2^{l[t]}$. Whenever
% a carry is generated, the corresponding component of $i$ is reset to zero. Our
% results show that \textit{opt3} accelerates the version of
% \textit{hierarchization} optimized with \textit{opt1} and \textit{opt2} up 1.5
% times. Note that \textit{opt3} introduces dependencies between the iterations of
% the innermost loop.
% 
% In our fourth optimization, \textit{opt4}, we manually unroll the $k$ loop in
% order to allow for vectorization in the implicit loops from the lines 6, 9, and
% 10. This optimization is incompatible with \textit{opt3} since \textit{opt3}
% makes the iterations of the $k$ loop dependent. Therefore, we apply this
% optimization only on top of \textit{opt1} and \textit{opt2}. For vectorization,
% we use in \textit{opt4} SSE intrinsics for integer operations. We show in
% Sec.~\ref{sec:evaluation} that the benefit of \textit{opt4} depends highly on
% the compiler used. In general, our results show that \textit{opt3} performs
% better than \textit{opt4}.

\begin{algorithm}[tbp]
\small{
	\caption{Evaluation.}
	\label{alg:evaluation}
	\begin{algorithmic}[1]
 		\Procedure{EvaluateKernel}{$w$}
    		\For{$j \leftarrow \Call{ChunkStart}{w} + lane; j < \Call{ChunkEnd}{w}; j \leftarrow j + 32$} 
    		\State $\textit{idx23} \leftarrow 0$
				\For{$g=1$ to $L$}
					\For{$b=1$ to $a(D, g)$}
						\State $\textit{idx1} \leftarrow 0$
						\State $\textit{p} \leftarrow 1$
						\If{$threadIdx.x = 0$}
							\State Compute $l$ for which $\text{pos}(l) = b$
						\EndIf
						\For{$t=1$ to $D$}
							\State $i \leftarrow \lfloor x[j][t] / 2^{l[t]} \rfloor \cdot 2 + 1$
							\State $p \leftarrow p \cdot \max(1 - |2^{l[t]} \cdot x[j][t] - i|, 0)$
							\State $\textit{idx1} \leftarrow \textit{idx1} \cdot 2^{l[t]} + \lfloor	x[j][t] / 2^{l[t]} \rfloor$
						\EndFor
						\State $r[j] \leftarrow r[j] + \textit{asg1d}[\textit{idx1}	+\textit{idx23}] \cdot p$ 
						\State $\textit{idx23} \leftarrow \textit{idx23} + 2^g$
					\EndFor
				\EndFor
    		\EndFor
    	\EndProcedure
 	\end{algorithmic}
}
\end{algorithm}

We move now to the algorithm for evaluation. The coordinates of all $N$ points
to be evaluated are stored contiguously in a matrix $x[N][D]$ which is processed
in chunks by each warp, each warp computing one chunk of size $w$. The memory
access pattern for one thread inside of its chunk is the same as for
$hierarchization$ algorithm where each thread computes points stored at indices
multiple of 32 starting from the corresponding lane. The $g$ and $b$ loops
traverse \textit{groups} and \textit{blocks} respectively. $\textit{idx23}$ is
used to index the beginning of the current \textit{block} in $\textit{asg1d}$,
while $idx1$ idexes the current point in this \textit{block}.

We start with optimizations for a better vectorization. For the CPU version,
in \textit{vec1} we use loop unroll-and-jam, loop tiling and we vectorize the
innermost loop $t$. Software pipelining can be done in the unrolled loop, but
$x$ is accessed with stride-$D$ instead of stride-1. Therefore, we change the
layout of $x$ such that $x[j][t] \rightarrow \tilde{x}[(j/\textit{m2}) \cdot D +
t][j \bmod \textit{m2}]$, where $\textit{m2}$ is the size of the tile. Now, a
stride-1 access on $\tilde{x}$ is possible which makes SSE vectorization
efficient. For GPUs, the observation we make is that the loop over dimensions on
line 11 has a non-contiguous access pattern for a warp because at each read
instruction the threads within a warp access elements found at $D$ distance.
\textit{vec1'} improves this by using a technique similar to the one we
described for both \textit{hierarchization} and \textit{evaluation} algorithms
for increasing the throughput through memory coalescing by threads within a
warp. More specific, \textit{vec1'} transposes the matrix $x$ such that instead
of storing the the points horizontally, one point per row, it stores them
vertically in $\tilde{x}[(N/32+1) \cdot D][32]$. Thus, all threads in a warp
access one row in this new matrix for each iteration over dimensions.

Regarding memory locality, \textit{ichg2} does a loop interchange by moving the
loop over interpolation points from line 2 inside the loop over \textit{blocks}
from line 5. In the case of GPUs, we have two variations of this optimization.
The first one, \textit{ichg2'} corresponds to \textit{ichg2} while the second
one, \textit{ichg2''}, additionally moves the loops over \textit{groups}
and \textit{blocks} outside the kernel. The goal of these optimizations is to
improve read accesses by having the innermost loop as tight as possible over the
block of memory.

The last optimization is a combination of strength reduction and loop invariant
code motion introduced in order to speed up integer operations. \textit{sred1}
precomputes all divisions in lines 12-14 outside the innermost loop and replaces
them with multiplications with the inverse. In addition, $idx1 \leftarrow
idx1 \cdot 2^{l[t]} + \lfloor x[j][t] / 2^{l[t]} \rfloor$ from line 14 is
replaced with $idx1 \leftarrow idx1 + \lfloor x[j][t] / 2^{l[t]} \rfloor
\cdot 2^{prefix\_sums[t + 1]}$, where $prefix\_sums[p] = \sum_{p=t}^{D-1}l[p]$.
This makes indexing a normal reduction and increases the instruction level
parallelism.
% 
% \begin{algorithm}[tbp]
% \small{
%  \caption{Evaluation.}
%  \label{alg:evaluation}                       
%  \begin{algorithmic}[1]
%     \FOR{$j=1$ to $m$}
% 		\STATE $\textit{idx23} \leftarrow 0$
% 		\FOR{$g=1$ to $n$}
% 			\FOR{$b=1$ to $a(d, g)$}
% 				\STATE $\textit{idx1} \leftarrow 0$
% 				\STATE $\textit{p} \leftarrow 1$
% 				\STATE Compute $l$ for which $\text{pos}(l) = b$
% 				\FOR{$t=1$ to $d$}
% 					\STATE $i \leftarrow \lfloor x[j][t] / 2^{l[t]} \rfloor \cdot 2 + 1$
% 					\STATE $p \leftarrow p \cdot \max(1 - |2^{l[t]} \cdot x[j][t] - i|, 0)$
% 					\STATE $\textit{idx1} \leftarrow \textit{idx1} \cdot 2^{l[t]} + \lfloor x[j][t] / 2^{l[t]} \rfloor$
% 				\ENDFOR
% 				\STATE $r[j] \leftarrow r[j] + \textit{asg1d}[\textit{idx1} + \textit{idx23}] \cdot p$
% 				\STATE $\textit{idx23} \leftarrow \textit{idx23} + 2^g$
% 			\ENDFOR
% 		\ENDFOR
%     \ENDFOR
%  \end{algorithmic}
% }
% \end{algorithm}
% 
% We now move to Alg.~\ref{alg:evaluation} representing the core of our
% \textit{evaluate} routine. We give a brief explanation of this algorithm. The
% $j$ loop from line 1 iterates over the points where we evaluate the sparse grid.
% These points are stored in the matrix $x[m][d]$. We store the evaluation results
% in the $r$ vector. Similarly to the algorithm for hierarchization, the $g$ and
% $b$ loops traverse \textit{groups} and \textit{blocks} respectively. For every
% point represented as a row of $x$, we use exactly one value from every $block$
% (line 12). $\textit{idx23}$ is used to index the beginning of the current
% \textit{block} in the 1d representation of the sparse grid, i.e. the 1d array
% $\textit{asg1d}$.
% 
% \textit{opt5} applied to this algorithm makes some assumptions about the number
% of iterations of the loops. We know that in general $\textit{asg1d}$ can be up
% to three orders of magnitude bigger than $x$. On the other hand, $2^g$ is
% typically smaller than the size of $x$. In this context, for better locality we
% permute the loops from $(j, g, b, t)$ to $(g, b, j, t)$. Note that this
% optimization also reuses $l$ from line 6 for multiple points from $x$. Our
% results show that this improves the basic version of evaluation up to a factor
% of 2.5. It is worth mentioning that this speedup depends on the number of
% iterations of the loops. The benefits of \textit{opt5} can be more impressive if
% we increase the ratio between the size of $\textit{asg1d}$ and the size of $x$.
% 
% Even after applying \textit{opt5}, there is not too much potential for
% vectorizing the $t$ loop due to the existence of dependencies between
% iterations. Nevertheless, we build \textit{opt6} on top of the permutation from
% \textit{opt5}. We can vectorize the innermost loop, $t$, by following the steps:
% grabbing iterations from the $j$ loop, inserting them in the $t$ loop, and
% reordering the instructions such that the same instructions operating on
% different data are packed together. The first two steps are equivalent to loop
% tiling, loop interchange, and loop unrolling. The third step is software
% pipelining in the unrolled loop. This is not enough as the access to $x$ is not
% stride-1. Actually, we have a stride-$d$ access. To cope with this problem, we
% change the layout of the $x$ matrix using the transformation $x[j][t]
% \rightarrow \tilde{x}[(j/\textit{m2}) \cdot d + t][j \bmod \textit{m2}]$, where
% $\textit{m2}$ is the size of the tile from the loop tiling transformation. Using
% $\tilde{x}$ instead of $x$ we have the stride-1 access required for efficient
% vectorization. By vectorizing the \textit{evaluate} routine using SSE
% intrinsics, we accelerate the version optimized using \textit{opt5} up to 2.6
% times.
% 
% \subsection{Parallelization of Sparse Grid Operations}
% 
% Our sparse grids routines are parallelized for shared memory machines using
% OpenMP. We provide in this part a brief description of the parallelization
% scheme used in \textit{fastsg}.
% 
% Alg.~\ref{alg:hierarchization} does not offer many possibilities for
% parallelization mainly due to the sequential optimizations. The $t$ loop has
% dependent iterations. The iterations of the $k$ loop are also dependent after
% applying \textit{opt3}. Additionally, we cannot update two \textit{groups} in
% parallel as one of them can depend on the old values of the second. Remember
% that we also aim at minimal memory consumption, meaning we want to keep our
% algorithm in-order, i.e. no extra copies of the data. Considering all these
% restrictions, the only loop whose iterations can be distributed among worker
% threads is the $b$ loop.
% 
% Regarding the evaluation algorithm, we have more options for parallelization. A
% first parallelization approach is taken to cover the case when the sparse grid
% is evaluated at a small number of points, e.g. one point. It implies
% distributing the iterations of the $b$ loop. In the case of one point, at the
% end of the \textit{evaluate} routine, the results computed by multiple threads
% are summed up. We use a coarser parallelization method if the number of points
% where we evaluate the sparse grid is reasonably big, e.g. bigger than 10,000. In
% this scenario, we parallelize the $j$ loop.
