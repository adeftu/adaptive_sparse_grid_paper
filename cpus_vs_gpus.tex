\section{CPUs versus GPUs}
Heterogeneous systems containing CPUs and accelerators allow us to reach higher
computational speeds while keeping power consumption at acceptable levels. The
most common accelerators nowadays, GPUs, are very different compared to
state-of-the-art general-purpose CPUs. While CPUs incorporate large caches and
complex logic for out-of-order execution, branch prediction, and speculation,
GPUs contain significantly more floating point units. They have in-order cores
which hide pipeline stalls through interleaved multithreading, e.g. allowing up
to 1536 concurrent threads per core\footnote{In Nvidia terminology a core is
called Streaming Multi-Processor.}.
Garland et al.~\cite{garland2010} refer to CPUs as latency oriented processors
with complex techniques used for extracting Instruction Level Parallelism (ILP)
from sequential programs. In contrast, GPUs are throughput oriented, containing
a large number of cores (e.g. 16) with wide SIMD units (e.g. 32 lanes), making
them ideal architectures for vectorizable codes. All applications can be run on
CPUs but only a subset can be ported to or deliver good performance on GPUs,
making them special purpose processors. In the following, we refer to GPUs and
CPUs as processors, but of different type.

% Given their specific architectural characteristics, programming the CPU and the GPU is inherently different. 
Programming the CPU and the GPU is inherently different. Multi-core CPUs are
programmed using threads through pthreads or OpenMP. For GPU programming, CUDA
is also based on threads, but there are differences. For synchronization, CUDA
only provides barriers within thread groups running on the same GPU core, and
atomic operations. For performance, the architectural details of GPUs have to be
considered. Maximizing the number of threads running concurrently on the GPU,
coalescing accesses to global memory, eliminating bank conflicts, minimizing the
number of branches, and utilizing the various memories appropriately (global,
shared, texture, constant) are important GPU optimizations.
% In contrast, CPU optimizations include blocking at different memory levels and arranging and aligning data for SSE / AVX operations.
In contrast, CPU optimizations include cache blocking and vectorization.

\subsection{GPU Architecture}

In contrast to CPUs which use the die space for complex control logic and large
caches, GPUs devote a higher percentage of transistors to floating point units.
GPUs provide massive parallelism and deliver better performance than CPUs
especially for applications with regular access patterns, e.g. dense matrix
operations \cite{vol2008}.

In the following we focus on the GeForce GTX 480 model of Nvidia Fermi.
This is a high-end GPU which contains 15 32-way SIMD units called by Nvidia
Streaming Multiprocessors (SM). In Nvidia terminology, each way is called a
Scalar Processor (SP). Each thread is generally executed on one of the 32 ways
of the SIMD unit. One of the main characteristics of Nvidia GPUs is multithreading
which offers the possibility to hide the latency originating from various
instructions, especially the latency caused by loads from and stores to RAM, by
executing a large number of threads concurrently with a low cost per context
switch.

Nvidia GPUs are SIMD based architectures. Each SM control unit creates, manages
and executes synchronously the threads in groups of 32 called warps. Every
instruction is broadcasted synchronously to all the active threads in a warp.
Branching may cause threads in the same warp to follow different execution
paths. This type of behavior of the threads inside a warp is called diverge. It
has the potential to severely reduce the performance of a GPU application, up to
a factor of 32.

GPUs have their own dedicated RAM, global memory, which is in the order of
several Gigabytes, e.g. GeForce GTX 480 has 1.5 GB of DDR3 memory. For
performance, GPUs are also equipped with multiple fast memories on the chip:
constant cache, texture cache and shared memory \cite{ryo2008}. The properties
of these fast memories vary in terms of latency, bandwidth and usage. The
constant cache is a read-only cache and, according to \cite{won2010}, its
level-1 has the lowest latency among the memories on the GPU. The texture cache
is a read-only memory used for optimizing the bandwidth rather than latency,
i.e. its latency is comparable with the one of the global memory. Finally, the
shared memory is a low latency, read-write memory, is private per SM and
controlled explicitly by the programmer.


\subsection{GPU Programming}

The Compute Unified Device Architecture (CUDA) is one of the available
frameworks for programming Nvidia GPUs. From a programming point of view, a CUDA
application has a CPU and a GPU part. The main responsibilities of the CPU part
are: allocating memory  on the GPU, transferring data to and from the GPU over
PCI Express and launching the GPU program called kernel. A kernel cannot contain
recursive functions. Each instance of a kernel in execution is a thread. Besides
being packed in warps, the threads are also grouped in blocks of threads. This
grouping is important as only the threads inside a block can synchronize via
barriers (\emph{\_\_syncthreads}) and share data from the shared memory. Each
thread has a 2d block identifier (\emph{blockIdx}) and a 3d thread identifier
(\emph{threadIdx}). Combining the block and the thread identifiers offers the
means to uniquely identify a thread on the GPU and to assign its part from the
workload to be computed on the GPU.

A first optimization for GPU programs consists of the proper exploitation of
multithreading. This is equivalent to maximizing the number of active threads
and can be achieved by reducing the register file and shared memory consumption
per thread. Second, efficient use of the memory hierarchy can provide a
substantial speedup to GPU applications. This optimization includes at least:
enabling coalesced accesses to global memory in order to reduce the number of
memory transactions, using the fast memories on-chip and reducing the number of
bank conflicts caused by accessing the shared memory. Third, divergent branches
can serialize the execution of the threads composing a warp. To improve the
performance, the number of divergent branches needs to be minimized. Note that
these are the optimizations which proved to be relevant to our application and
they represent only a subset of the optimizations applicable to CUDA
applications. For a detailed list of optimizations we refer the reader to
\cite{cuda}.

% The number of active threads is not the same for all applications. An optimization direction for GPU programs consists in maximizing the number of active threads. This is typically achievable by reducing the register and shared memory consumption per thread.

% Nvidia GPUs have a Single Instruction Multiple Thread (SIMT) architecture. The SM's control unit creates, manages and executes synchronously the threads in groups of 32 called warps. Every instruction is broadcast synchronously to all the active threads in a warp. In this context, divergence caused by branches may serialize the execution of the threads composing a warp. To improve the performance, the number of divergent branches needs to be minimized. 

% GPUs have multiple fast memories on the chip: constant cache, texture cache and shared memory\cite{ryo2008}. Exploiting the memory hierarchy properly is another direction for optimizing CUDA programs.

% GPUs have their own dedicated memory which is in the order of several Gigabytes, e.g. Tesla has 4 GB of memory. Minimizing the memory consumption is also an optimization that should be considered by a CUDA developer.

% Recursive function are not allowed on GPUs\cite{cuda}. 