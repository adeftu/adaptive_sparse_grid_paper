\section{CPUs and GPUs}
\label{sec:cpus_vs_gpus}
Heterogeneous systems containing CPUs and accelerators allow us to reach higher
computational speeds while keeping power consumption at acceptable levels. The
latest developments show a trend towards an unification in terms of hardware
features of both architectures. Instead of focusing only on the classic way to
increase the speed of CPUs by raising the frequency with each new release, the
manufacturers are now trying to exploit other directions as well, such as
vectorization or many-cores. An eloquent example is Intel's Many Integrated
Core Architecture (MIC) \cite{intel_mic} which is announced to include 512-bit
wide vector registers and more than 50 cores per chip (in Knights Corner product
line). On the other hand, GPUs manufacturers started to include CPU-like
features in their products. Nvidia's Fermi \cite{fermi} was the first GPU
generation from Nvidia featuring a cache coherent memory and today all devices
tend to have a cache hierarchy similar to the one on CPUs.

Nonetheless, when performance becomes a priority, programming the GPUs requires
being aware of some architectural features that are different. We will describe
in the following subsections the current GPU architectures using CUDA terminology and
provide some brief guidelines for programming Nvidia cards.

\subsection{GPU Architecture}

GPUs are heavily optimized for fast arithmetic operations and in consequence
have many dedicated floating point units. Therefore, not all applications are
suitable for running on GPUs, or at least not as efficient as on CPUs, but only
those that exhibit regular memory access pattern and require a great amount of
computational power.

The CUDA architecture from Nvidia is based on \textit{Streaming Multiprocessors}
(SM), which are SIMD cores with the role of creating, scheduling and executing
concurrent threads. One such multiprocessor is composed of several lanes, or
\textit{Scalar Processors} (SP), and has a dedicated shared memory for fast
access by the threads running on it. The thread-execution manager automatically
schedules threads on processors without the need from the programmer for writing
explicit code for this. The execution is done in groups of 32 threads, called
\textit{warps}, such that every instruction is broadcast synchronously to all
the active threads in a \textit{warp}. The degree of parallelism is high when
all these 32 threads execute the same instruction and very low when thread
divergence caused by branching occurs. This is an important factor to consider
when writing code for GPUs with performance in mind.

Each thread has a small local private memory. Threads are grouped in \textit{blocks}. 
Each \textit{block} has a
shared memory accessible by all threads in the \textit{block}. Specifically
designed for graphics processing, the cards are also equipped with read-only,
fast, on-chip memories like constant and texture caches which are used for
different purposes. For example the first one is optimized for latency, while
the second one for bandwidth. Finally, there is also the global memory which has
the largest capacity, but is the slowest among all. Starting with the Fermi
architecture, a finer level of control is possible for configuring how the
allocations of these memories is done for the \textit{kernels}, giving the
programmer more flexibility, but with the cost of exposing details and
complexities of the underlying architecture.

\subsection{GPU Programming}

While SIMD code on CPUs can be written using assembly code, a high-level
language like C (with intrinsic functions), libraries such as Intel Array
Building Blocks or just relying on a compiler with auto-vectorizing
capabilities, Nvidia GPUs are usually  programmed through their dedicated
framework, called CUDA. CUDA allows the programmer to define functions
(\textit{kernels}) to be executed in parallel by many threads on different
processors. When the \textit{kernels} are launched on the GPU, a given topology
must be specified. Threads are grouped in \textit{thread blocks} which are laid
in a grid. Each \textit{thread block} has a unique 2d index which gives its
position in the grid, while each thread has a 3d index identifying the thread in
a \textit{block}. This aspect allows uni-, bi-, and three-dimensional
topologies, the right configuration being chosen depending on the problem to be
solved. Threads in a \textit{block} are scheduled to be run on a single SM, fact
that allows only the synchronization among the threads from a single
\textit{block} using the \emph{\_\_syncthreads} barrier. Also, these threads can
share data among themselves through the shared memory of the SM.

Usually when choosing to write code for GPUs, performance is the most important
reason for this decision. Optimizations address mostly thread scheduling and
memory access. In the first category the most important ones are choosing the
right topology when launching the \textit{kernel}, thus maximizing the number of
active threads per SM, and trying to avoid or at least minimize the number of
divergent branches that could lead to a serial execution of threads composing a
\textit{warp}. Regarding memory hierarchy, it is preferable to use the faster 
on-chip memories whenever possible, to reduce the number of bank conflicts when
accessing the shared memory and to reduce the number of memory transactions to
main memory through coalescing. We applied these optimizations to our
application, but for a detailed list we refer the reader to \cite{cuda}.
